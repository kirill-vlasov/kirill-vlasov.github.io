\documentclass{article}
\usepackage[utf8]{inputenc}
\usepackage[russian]{babel}

\usepackage{xcolor}
  \definecolor{yb-blue}{RGB}{5,60,94}
\usepackage{hyperref}
  \hypersetup{colorlinks=true,urlcolor=yb-blue,pdfborder={0 0 0}}

\usepackage{ragged2e}
\usepackage[a4paper, total={6in, 8in}]{geometry}


\begin{document}
\pagestyle{empty}
\setlength{\topskip}{0mm}
\setlength{\parindent}{0pt}
\setlength{\parskip}{4pt}
\raggedright
\interfootnotelinepenalty=10000


\justifying
  
\section*{Кирилл Власов}

\textbf{Java Developer \& Architect}\\%
\href{mailto:mail@kvlasov.ru}{mail@kvlasov.ru}\\%
\href{https://ru.linkedin.com/in/kirill-vlasov-48451427}{LinkedIn}
\href{https://github.com/kirill-vlasov}{GitHub}\\%
+7 (912) 77 000 80	

\vspace{1em}

\subsection*{CTO \& Java Developer}


\section*{Skills}
Java, Spring, Hibernate,
Linux, Docker, Bash, Python,
Git, Maven, Gradle,
Postgresql, Couchdb


\subsection*{\href{https://www.quickresto.ru/}{QuickResto}}
\smallЯнварь 2013 - по настоящее время
\normalsize

Проектирование и разработка сервиса автоматизации предприятий общественного питания и ресторанной деятельности. Проект начинался как startup компании EDGEX Software, впоследствии переросший в самостоятельный бизнес и в новую компанию - QuickResto LLC.

\begin{itemize}
  \item Разработка основных прикладных модулей серверного приложения
  \item Разработка модулей учета себестоимости и остатков продукции
  \item Интеграция с ЭВОТОР
  \item Проектирование и разработка работы приложения для франшиз
  \item Организация работы команды программистов по технической части, выбор стиля разработки (agile планирование)
  \item Планирование, организация и управление небольшой командой разработчиков (5-10 человек)
\end{itemize}

\subsection*{QuickResto DevOps}
\smallЯнварь 2013 - по настоящее время
\normalsize

Проектирование и разработка инфраструктуры работы и обслуживания приложений QuickResto на собственных серверах (Dell PowerEdge), где в качестве системы виртуализации используется Xen. Приложения работают в процессах контейнерной виртуализации docker.
\begin{itemize}
  \item Была спроектирована система безостановочного апгрейда приложений по типу blue-green.
  \item Разработал схему взаимодействия программных сервисов для безотказной работы приложений (failover).
  \item Организовал работу отдела эксплуатации production серверов и контроль за результатами отдела.
  \item Организовал работу сервисов для внутреннего использования (JIRA, git, jenkins, openvpn, xen VMs и т.д.).
\end{itemize}


\subsection*{EDGEX Software}
\smallИюнь 2011 - Январь 2013
\normalsize

Проектирование, архитектура и разработка фреймворка для построения приложений с быстрым прототипированием. Платформа позволяет представить web приложение в multi-tier архитектуре и проводить разработку программных компонент отдельно друг от друга. В качестве базы данных приложения используется PostgreSQL СУБД.
Приложение на EDGEX платформе готово к запуску в облачной инфраструктуре.
Дополнительно можно разрабатывать терминальные приложения, в основе протокола взаимодействия находится документоориентированная база данных Apache CouchDB. 
Для приложения возможна поддержка multitenancy.

\begin{itemize}
  \item Создание платформы и фрейворка приложений
  \item Организация синхронной очереди исполнения задач и построение на ее основе взаимодействия с CouchDB терминалами
  \item Разработка сервера запуска приложений
  \item Проектирование thin web клиента (html + javascript)
  \item Организация continuous integration и continuous delivery процессов тестирования и развертывания приложений в клиентских слоях
\end{itemize}


\subsection*{\href{https://www.compassplus.ru/}{Compass Plus}}

\subsection*{Розничная банковская система TranzAxis}
\smallИюнь 2011 - Январь 2013
\normalsize
Участие в разработке комплекса приложений розничных банковских систем (retail banking system) на платформе RadixWare. Работа с этой платформой имеет много общего с разработкой на Java EE. 
\begin{itemize}
  \item Разработка серверного приложения как WSDL сервис для работы с банкоматами (NDC/DDC протоколы) и POS терминалами (TPTP/TITP/ISO8583 протоколы).
  \item Проектирование и разработка приложений для мониторинга и управления банковскими терминалами.
  \item Разработка приложения для формирования алгоритмов, сценария и экранов работы банкоматного ПО.
  \item Участие в разработке компонентов протоколов ISO8583, TIC и других модулей взаимодействия банковских систем.
  \item Проектирование и разработка графических интерфейсов приложения на Qt (QtJambi).
\end{itemize}

\subsection*{Терминальное ПО Kalignite ATMs}
\smallФевраль 2010 - Июнь 2011
\normalsize

Разработка программного обеспечения для банкоматов и киосков на платформе Kalignite под архитектуру CEN/XFS приложений.
\begin{itemize}
  \item Проектирование и разработка взаимодействия банкоматного ПО с серверным приложением.
  \item Конструирование алгоритма и сценариев работы банкомата, экранов пользователя и супервизора банкомата.
  \item Разработка программ мониторинга и обслуживания банкоматов (обновление терминального ПО, журналирование и выгрузка журналов на сервер, система нотификации состояния банкомата).
  \item Взаимодействие с иностранными представителями компании ПО Kalignite.
\end{itemize}

Проект прошел успешную инсталляцию в банке "Нейвабанк" (г. Екатеринбург) и в проекте “Мультикарта” (банк ВТБ24) на банкоматах фирмы Nautilus.


\subsection*{TranzWare Online} 
\smallМай 2007 - Январь 2010
\normalsize
\begin{itemize}
  \item Разработка и сопровождение серверного ПО для обработки запросов от банкоматных приложений (протоколы NDC/DDC).
  \item Разработка модулей и компонент для сервиса мониторинга банкоматов и банковских POS терминалов.
  \item UI компоненты и интерфейс настроек банковских терминалов по стандартам NDC/DDC.
\end{itemize}


\subsection*{TranzWare Mobile}
\smallИюнь 2004 - Май 2007
\normalsize
\begin{itemize}
  \item Мобильный банкинг: разработка приложения дистанционного управления банковским счетом для мобильных устройств на платформе Java ME; серверное приложение для работы с мобильным клиентом по протоколу FIMI.
  \item MobiCash ATMs: сопровождение программного обеспечения для банкоматов MobiCash.
\end{itemize}


\end{document}


